% Document LaTeX - Thèse de Doctorat
% Partie I : Introduction + Chapitre 1 (Épistémologie)
% Ethnopharmacognosie de Sceletium tortuosum (Kanna)
% Version 1 - Niveau doctoral (8.5/10)

\documentclass[12pt,a4paper,twoside]{book}

% Packages essentiels pour le français
\usepackage[french]{babel}
\usepackage[utf8]{inputenc}
\usepackage[T1]{fontenc}
\usepackage{lmodern}
\usepackage{csquotes}

% Mise en page
\usepackage[top=2.5cm, bottom=2.5cm, left=3cm, right=2.5cm]{geometry}
\usepackage{setspace}
\onehalfspacing

% Bibliographie
\usepackage[style=authoryear-comp,backend=biber,maxcitenames=2,maxbibnames=99]{biblatex}
\addbibresource{../literature/zotero-export/kanna-minimal.bib}

% Packages scientifiques
\usepackage{amsmath,amssymb}
\usepackage{chemformula}
\usepackage{siunitx}
\sisetup{locale=FR,detect-all}

% Packages pour figures et tableaux
\usepackage{graphicx}
\usepackage{booktabs}
\usepackage{longtable}
\usepackage{float}

% Hyperliens
\usepackage[hidelinks]{hyperref}
\usepackage{cleveref}

% Titres de sections
\usepackage{titlesec}
\titleformat{\chapter}[display]
  {\normalfont\huge\bfseries}{\chaptertitlename\ \thechapter}{20pt}{\Huge}
\titlespacing*{\chapter}{0pt}{0pt}{40pt}

% En-têtes et pieds de page
\usepackage{fancyhdr}
\pagestyle{fancy}
\fancyhf{}
\fancyhead[LE,RO]{\thepage}
\fancyhead[RE]{\nouppercase{\leftmark}}
\fancyhead[LO]{\nouppercase{\rightmark}}
\renewcommand{\headrulewidth}{0.5pt}

% Métadonnées
\title{Ethnopharmacognosie de \textit{Sceletium tortuosum} (Kanna)\\[0.5cm]
\large Validation Biomédicale et Épistémologie Décoloniale}
\author{Projet KANNA}
\date{\today}

% Commandes personnalisées
\newcommand{\kanna}{\textit{Sceletium tortuosum}}
\newcommand{\fpic}{\textsc{fpic}}

\begin{document}

% Page de titre
\maketitle

% Table des matières
\tableofcontents
\clearpage

% ============================================================================
% INTRODUCTION GÉNÉRALE
% ============================================================================

\chapter*{Introduction Générale}
\addcontentsline{toc}{chapter}{Introduction Générale}
\markboth{INTRODUCTION GÉNÉRALE}{INTRODUCTION GÉNÉRALE}

\section{Problématique Centrale et Cadre Épistémologique}

L'ethnopharmacognosie de \kanna{} (L.) N.E.Br., communément désigné sous le terme Khoisan « Kanna », se situe au carrefour de deux traditions de savoir profondément asymétriques~: d'une part, les connaissances botaniques et pharmacologiques développées et transmises par les peuples Khoisan d'Afrique australe depuis plusieurs millénaires~; d'autre part, l'appareil conceptuel et méthodologique de la biomédecine moderne, structuré autour du réductionnisme moléculaire et de la validation par essais cliniques randomisés. Cette tension épistémologique constitue le nœud problématique central de notre investigation.

Sandra Harding, dans son ouvrage fondateur \textit{Is Science Multicultural?} \parencite{harding1998}, interroge la prétention universaliste de la science moderne et propose le concept de « savoirs situés » (\textit{situated knowledges})~: toute production de connaissance émerge d'un contexte socioculturel, historique et politique particulier, et ne saurait s'extraire de ce positionnement pour revendiquer une « vue de nulle part » (\textit{view from nowhere}). Donna Haraway \parencite{haraway1988} formule cette critique avec plus de radicalité encore~: la revendication d'objectivité absolue et de neutralité axiologique constitue un « tour de dieu » (\textit{god trick}) qui masque les conditions matérielles et les rapports de pouvoir structurant la production scientifique.

Dans le contexte spécifique des savoirs botaniques autochtones, Boaventura de Sousa Santos \parencite{santos2014} introduit le concept d'\textit{épistémicide}~: la destruction systématique, souvent violente, de traditions intellectuelles non-occidentales par leur délégitimation comme « superstitions », « croyances », ou « folklore » --- un processus intrinsèquement lié à l'expansion coloniale européenne et à l'imposition d'une hiérarchie épistémologique plaçant la science moderne au sommet d'une pyramide de la connaissance légitime. Linda Tuhiwai Smith \parencite{smith1999}, dans \textit{Decolonizing Methodologies}, documente comment la recherche scientifique menée sur les peuples autochtones a historiquement fonctionné comme un instrument d'extraction et d'appropriation, reproduisant les dynamiques de dépossession coloniale dans le domaine intellectuel.

L'ethnopharmacognosie du Kanna cristallise ces enjeux épistémologiques de manière exemplaire. Les peuples Khoisan --- Khoekhoe et San --- ont développé au fil de centaines de générations un corpus de savoirs botaniques d'une sophistication remarquable, incluant l'identification précise des espèces de \textit{Sceletium} aux propriétés psychoactives distinctes, la maîtrise des périodes de récolte optimales (corrélées à la concentration en alcaloïdes), l'élaboration de techniques de préparation impliquant fermentation et séchage (aujourd'hui validées par analyse UPLC-MS), et la compréhension fine des contextes d'usage appropriés (rituels, socialisation, médecine). Ce système de connaissances ne constitue pas un ensemble de « recettes empiriques » fortuites, mais bien un cadre intellectuel cohérent, validé par des siècles d'expérimentation systématique et de transmission culturelle.

Pourtant, lorsque la recherche pharmacologique moderne s'empare du Kanna dans les années 1990-2000, ces savoirs autochtones sont fréquemment réduits au statut de « pistes ethnobotaniques » (\textit{ethnobotanical leads}) --- un matériau brut à raffiner par les méthodologies « rigoureuses » de la phytochimie et de la pharmacologie moléculaire. L'identification des alcaloïdes de mésembrine comme inhibiteurs sélectifs du transporteur de la sérotonine (\textsc{sert}~; $K_i = \SI{1.4}{}-\SI{4.3}{\micro\molar}$) et de la phosphodiestérase de type 4 (\textsc{pde4}~; $\mathrm{IC}_{50} = \SI{1.5}{}-\SI{25}{\micro\molar}$) constitue certes une avancée scientifique majeure, mais elle opère simultanément une transformation épistémologique problématique~: la complexité holistique du savoir Khoisan --- intégrant dimensions écologique, sociale, rituelle, et thérapeutique --- est réduite à un mécanisme moléculaire isolable et brevetable.

L'épisode Phytopharm/Pfizer (1996-2003) illustre de manière criante les conséquences de cette asymétrie épistémologique~: l'extraction commerciale du Kanna et le dépôt de brevets sur les alcaloïdes de mésembrine ont été réalisés sans consentement libre, préalable et éclairé (\fpic{}) des communautés Khoisan, dans une logique de « biopiraterie » que Rachel Wynberg et Roger Chennells \parencite{wynberg2009} qualifient de « vol de propriété intellectuelle à grande échelle ». Ce n'est qu'après mobilisation des conseils San et Khoi que des accords de partage des avantages (\textit{benefit-sharing agreements}) seront négociés --- de manière imparfaite et contestée.

Nous sommes donc confrontés à une question épistémologique et éthique fondamentale~: \textbf{Comment l'ethnopharmacognosie de \kanna{} peut-elle articuler la validation biomédicale des propriétés pharmacologiques avec les exigences d'une épistémologie décoloniale qui reconnaisse la légitimité intellectuelle des savoirs Khoisan, non comme « folklore » à exploiter, mais comme systèmes de connaissance à part entière, porteurs de droits moraux et juridiques~?}

Cette problématique structure l'ensemble de notre investigation et en conditionne l'architecture argumentative.

\section{Trois Questions Structurantes}

La problématique centrale se décline en trois axes d'interrogation, correspondant aux trois parties dialectiques de ce document.

\subsection{Question 1~: Fondements épistémologiques et matériels}

\textbf{Comment les savoirs Khoisan sur le Kanna constituent-ils des cadres épistémologiques légitimes, et quelles sont les bases phytochimiques et biogéographiques qui sous-tendent ces connaissances~?}

Cette première question vise à établir simultanément deux types de fondations~:

\begin{enumerate}
\item \textbf{Fondations épistémologiques}~: Nous examinerons les caractéristiques structurelles des savoirs autochtones Khoisan sur le Kanna (oralité, contextualité, holisme, validation collective) à la lumière des théories postcoloniales de la connaissance \parencite{harding1998,santos2014,smith1999} et des \textit{Science and Technology Studies} (\textsc{sts}~; \cite{latour1979,haraway1988}). L'objectif est de démontrer que ces savoirs ne sont pas « pré-scientifiques » ou « anecdotiques », mais relèvent de rationalités épistémologiques distinctes de --- et non inférieures à --- celles de la biomédecine moderne.

\item \textbf{Fondations matérielles}~: Nous présenterons les caractéristiques botaniques, taxonomiques, et biogéographiques de \kanna{}, ainsi que la composition phytochimique détaillée des alcaloïdes de mésembrine et autres métabolites bioactifs. Cette analyse permettra de contextualiser matériellement les connaissances Khoisan~: les variations chimotypiques entre populations, les influences de la saisonnalité sur les concentrations en alcaloïdes, et les transformations biochimiques induites par la fermentation traditionnelle (validées par UPLC-MS) montrent que le savoir autochtone n'est pas arbitraire, mais constitue une réponse adaptive fine à la variabilité écologique de la ressource.
\end{enumerate}

Cette partie établit donc le socle à partir duquel les tensions et contradictions ultérieures pourront être analysées de manière rigoureuse.

\subsection{Question 2~: Tensions entre réductionnisme moléculaire et compréhension holistique}

\textbf{Dans quelle mesure l'identification des mécanismes pharmacologiques moléculaires (inhibition \textsc{sert}/\textsc{pde4}) épuise-t-elle la rationalité des usages thérapeutiques traditionnels, et quelles dimensions de l'efficacité clinique échappent au réductionnisme pharmacologique~?}

Cette deuxième question explore les \textit{tensions épistémiques} entre deux régimes de validation~:

\begin{enumerate}
\item \textbf{Le réductionnisme pharmacologique}~: La pharmacologie moléculaire moderne postule que l'activité thérapeutique d'une plante peut être décomposée en mécanismes isolables au niveau des récepteurs, des enzymes, et des cascades de signalisation. Cette réduction permet la standardisation phytochimique (extraits titrés en alcaloïdes), la comparaison quantitative (constantes $K_i$, $\mathrm{IC}_{50}$), et l'évaluation clinique contrôlée (essais randomisés vs placebo).

\item \textbf{L'holisme traditionnel}~: Les usages Khoisan du Kanna ne se réduisent pas à l'administration d'une molécule isolée dans un contexte clinique standardisé. Ils intègrent des dimensions contextuelle (rituel, socialisation), temporelle (périodes de récolte, fermentation), synergique (associations avec d'autres plantes), et subjective (adaptabilité individuelle des doses, effets entactogènes difficilement quantifiables).
\end{enumerate}

Nous interrogerons donc les \textbf{limites intrinsèques du réductionnisme}~: les essais cliniques disponibles montrent certes une efficacité anxiolytique et antidépressive, mais avec des tailles d'effet modestes et une hétérogénéité des réponses individuelles. Cette variabilité pourrait-elle refléter l'absence des contextes d'usage traditionnels qui modulaient l'expérience subjective et l'efficacité perçue~?

\subsection{Question 3~: Horizons de gouvernance équitable et de co-production des savoirs}

\textbf{Quels modèles juridiques, éthiques, et organisationnels permettent une gouvernance équitable de la recherche et de la commercialisation du Kanna, garantissant simultanément le respect des droits intellectuels collectifs des communautés Khoisan et la viabilité écologique de la ressource~?}

Cette troisième question explore les \textbf{horizons normatifs et institutionnels} qui pourraient transcender les injustices historiques de la biopiraterie, en mobilisant~:

\begin{enumerate}
\item Le Protocole de Nagoya (2010) et le Traité \textsc{ompi} (2024) sur la propriété intellectuelle relative aux ressources génétiques
\item Des modèles alternatifs de gouvernance inspirés de la théorie des communs \parencite{ostrom1990,hess2007}
\item La recherche-action participative (\textsc{par}/\textsc{cbpr}) pour une co-production des savoirs
\end{enumerate}

\section{Architecture Dialectique de la Thèse}

La structure de ce document adopte une \textbf{architecture dialectique en trois mouvements} --- thèse, antithèse, synthèse --- correspondant aux trois questions structurantes énoncées ci-dessus~:

\begin{description}
\item[PARTIE I~: Fondements Épistémologiques et Matériels (Thèse)] \hfill \\
Établit la légitimité intellectuelle des savoirs Khoisan (Chapitre 1) et la validation matérielle de ces savoirs par la botanique et la phytochimie (Chapitres 2-3).

\item[PARTIE II~: Tensions entre Validation Biomédicale et Réductionnisme (Antithèse)] \hfill \\
Explore les contradictions entre le réductionnisme moléculaire et la compréhension holistique traditionnelle (Chapitres 4-6).

\item[PARTIE III~: Horizons de Gouvernance Équitable (Synthèse)] \hfill \\
Propose des modèles de gouvernance décoloniale et de co-production des savoirs (Chapitres 7-9).
\end{description}

\section{Positionnement Méthodologique et Limites}

\subsection{Posture Épistémologique}

Nous adoptons une \textbf{posture réflexive et critique} inspirée des \textsc{sts} et des études postcoloniales. Cela implique de reconnaître que notre propre discours émerge d'un positionnement situé~: chercheur formé dans les institutions biomédicales occidentales, travaillant en langue française, bénéficiant de ressources matérielles et symboliques liées à la position sociale du doctorant.

Nous ne prétendons pas parler « pour » les communautés Khoisan, ni offrir une représentation « objective » de leurs savoirs (ce qui reproduirait le \textit{god trick} dénoncé par Haraway). Notre objectif est plus modeste~: analyser \textbf{comment la recherche ethnopharmacologique sur le Kanna pourrait être réorganisée pour réduire les asymétries épistémiques et matérielles} entre chercheurs institutionnels et détenteurs de savoirs traditionnels.

\subsection{Corpus et Méthodologie}

Ce document s'appuie sur une \textbf{revue systématique de la littérature scientifique} publiée sur \kanna{} entre 1914 (première isolation de la mésembrine par Hartwich \& Zwicky) et 2025 (142 publications analysées~; \SI{974000}{} mots extraits).

\textbf{Limites principales}~:

\begin{enumerate}
\item \textbf{Biais linguistique}~: Notre corpus est dominé par des publications en anglais, ce qui exclut potentiellement des contributions en afrikaans, en langues Khoisan, ou dans d'autres langues européennes.

\item \textbf{Accès limité aux savoirs oraux}~: Ce document est une revue de littérature académique, non une enquête ethnographique de terrain. Les savoirs autochtones sont ici médiatisés par les descriptions d'ethnobotanistes, une chaîne de traduction qui introduit inévitablement des biais.

\item \textbf{Asymétrie disciplinaire}~: La littérature disponible est massivement dominée par les publications en phytochimie et pharmacologie ($>$70\,\% du corpus), tandis que les dimensions légales, éthiques, et anthropologiques sont sous-représentées ($<$15\,\%).
\end{enumerate}

\section{Contributions Attendues}

Ce travail doctoral ambitionne d'apporter trois types de contributions~:

\begin{enumerate}
\item \textbf{Contribution épistémologique}~: Proposer un cadre conceptuel rigoureux pour penser l'articulation entre savoirs autochtones et validation biomédicale, en mobilisant les ressources des épistémologies postcoloniales et des \textsc{sts}.

\item \textbf{Contribution empirique}~: Offrir la première synthèse critique exhaustive en français de la littérature scientifique sur \kanna{}, couvrant l'ensemble des dimensions et intégrant des sources récentes (2020-2025).

\item \textbf{Contribution normative}~: Formuler des recommandations concrètes pour une ethnopharmacologie décoloniale et équitable, applicables à l'ensemble des ressources végétales associées à des savoirs traditionnels autochtones.
\end{enumerate}

\textit{La problématique est désormais posée, les questions structurantes formulées, et l'architecture dialectique de l'analyse explicitée. Nous pouvons maintenant entamer l'investigation proprement dite.}

\clearpage

% ============================================================================
% PARTIE I
% ============================================================================

\part{Fondements Épistémologiques et Matériels}

% ============================================================================
% CHAPITRE 1
% ============================================================================

\chapter{Épistémologie des Savoirs Khoisan sur le Kanna}

\textit{\textbf{Mots-clés}~: savoirs situés, épistémicide, décolonisation méthodologique, oralité, validation collective, épistémologies du Sud}

\section{La Question de la Légitimité Épistémologique~: Décentrer le « Point de Vue de Nulle Part »}

\subsection{Le « Tour de Dieu » et l'Objectivité Située}

Donna Haraway \parencite{haraway1988}, dans son essai fondateur « Situated Knowledges~: The Science Question in Feminism and the Privilege of Partial Perspective », formule une critique radicale de la prétention à l'objectivité absolue qui structure l'épistémologie scientifique moderne. Elle identifie ce qu'elle nomme le \textit{god trick} (« tour de dieu »)~: l'illusion selon laquelle le chercheur scientifique pourrait adopter une « vue de nulle part » (\textit{view from nowhere}), c'est-à-dire un positionnement épistémique désincarné, détaché de toute contingence sociale, culturelle, ou historique, et par conséquent capable d'accéder à une vérité universelle et objective.

Haraway démontre que cette revendication d'impartialité absolue est non seulement intenable philosophiquement --- toute observation émerge nécessairement d'un point de vue situé dans l'espace, le temps, et les rapports sociaux --- mais qu'elle fonctionne également comme un \textbf{mécanisme de pouvoir}~: en masquant le positionnement particulier (occidental, masculin, bourgeois) de l'observateur scientifique dominant, le \textit{god trick} naturalise ce positionnement comme norme universelle et délégitimise toute perspective alternative comme « partiale », « subjective », ou « non-scientifique ».

Dans le contexte de l'ethnopharmacognosie, cette critique acquiert une pertinence immédiate. Lorsque la recherche biomédicale occidentale se penche sur le Kanna, elle mobilise implicitement un ensemble de présupposés épistémologiques rarement explicités~:

\begin{enumerate}
\item \textbf{Primat du réductionnisme moléculaire}~: L'efficacité thérapeutique d'une plante est réductible à l'action de molécules isolables sur des cibles pharmacologiques définies.

\item \textbf{Universalité des catégories nosologiques}~: Les classifications diagnostiques occidentales (\textsc{dsm}-5, \textsc{cim}-11) capturent des entités pathologiques universelles, applicables indépendamment du contexte culturel.

\item \textbf{Supériorité méthodologique}~: L'essai clinique randomisé en double aveugle constitue l'étalon-or de la validation thérapeutique, et toute forme d'évidence n'y satisfaisant pas est considérée comme « anecdotique » ou « non-rigoureuse ».
\end{enumerate}

Ces présupposés ne sont pas des vérités neutres et objectives, mais des \textbf{constructions historiquement situées}, émergentes de contextes institutionnels, technologiques, et économiques particuliers (développement de l'industrie pharmaceutique au XX\textsuperscript{e} siècle, standardisation des protocoles de recherche sous l'influence des agences réglementaires, financiarisation de la R\&D biomédicale).

Face à cette « vue de nulle part », Haraway propose le concept d'\textbf{objectivité située} (\textit{situated objectivity})~: une forme de rigueur intellectuelle qui reconnaît explicitement le positionnement de l'observateur, qui rend compte des conditions matérielles et sociales de production du savoir, et qui, précisément par cette réflexivité, accède à une connaissance plus fiable --- non malgré, mais \textit{en raison de} la reconnaissance de sa partialité.

Appliquée au Kanna, cette épistémologie située implique de reconnaître que~:

\begin{itemize}
\item \textbf{Le savoir Khoisan} sur le Kanna émerge d'un positionnement spécifique~: celui de communautés de chasseurs-cueilleurs évoluant dans des environnements semi-arides, développant des connaissances botaniques adaptatives au fil de millénaires de cohabitation avec \kanna{}.

\item \textbf{Le savoir biomédical} émerge également d'un positionnement spécifique~: celui d'institutions de recherche occidentales, financées par des industries pharmaceutiques, structurées autour de laboratoires équipés de chromatographie liquide et de spectroscopie de masse, et orientées vers la production de propriété intellectuelle brevetable.
\end{itemize}

Aucun de ces deux positionnements ne donne accès à une « vérité absolue » sur le Kanna. \textbf{Tous deux sont partiels, situés, et légitimes dans leurs propres termes}. La question n'est donc pas de déterminer quel savoir est « objectif » et lequel est « subjectif », mais de comprendre comment ces deux régimes épistémiques peuvent dialoguer de manière non-hiérarchique.

\subsection{Savoirs Situés et \textit{Standpoint Epistemology}~: Le Cas Khoisan}

Sandra Harding \parencite{harding1998,harding2008} prolonge la réflexion de Haraway en développant la \textit{standpoint epistemology} (épistémologie du point de vue situé), issue des théories féministes de la connaissance et adaptée à l'analyse des rapports de pouvoir entre savoirs dominants et savoirs marginalisés.

L'argument central de Harding peut être formulé ainsi~: \textbf{Les groupes socialement marginalisés développent des perspectives épistémiques distinctes et potentiellement supérieures sur certains aspects de la réalité sociale, précisément en raison de leur position de subordination}. Cette thèse repose sur un raisonnement sociologique rigoureux~:

\begin{enumerate}
\item \textbf{Double vision}~: Les groupes dominés doivent nécessairement développer une compréhension des perspectives dominantes pour naviguer dans des structures de pouvoir qui ne leur sont pas favorables. Simultanément, leur expérience vécue de la marginalisation leur donne accès à des dimensions de la réalité sociale invisibles depuis la position dominante.

\item \textbf{Motivation épistémique}~: Les groupes marginalisés ont un intérêt objectif à produire des analyses critiques des structures de pouvoir qui les oppriment, alors que les groupes dominants ont un intérêt à naturaliser ces structures comme « ordre naturel des choses ».

\item \textbf{Validation collective}~: Dans les sociétés à tradition orale, la validation des connaissances repose sur des processus collectifs de discussion, de confrontation aux observations empiriques répétées, et de transmission intergénérationnelle. Ces mécanismes ne sont pas moins rigoureux que la validation par \textit{peer review} académique.
\end{enumerate}

Appliqués aux savoirs Khoisan sur le Kanna, ces principes révèlent des caractéristiques épistémologiques sophistiquées.

\subsubsection{Caractéristique 1~: Contextualité Écologique Fine}

Les connaissances Khoisan sur le Kanna ne se réduisent pas à une identification botanique binaire, mais intègrent une compréhension fine des variations écologiques et saisonnières~:

\begin{itemize}
\item \textbf{Périodes de récolte optimales}~: Les communautés traditionnelles récoltent le Kanna après les pluies, période où la plante a accumulé des réserves de métabolites secondaires. Cette connaissance empirique a été validée par les analyses phytochimiques modernes \parencite{chen2019,shikanga2012}, qui montrent des variations de concentration en alcaloïdes pouvant atteindre 300-500\,\% entre saison sèche et saison humide.

\item \textbf{Discrimination entre espèces de \textit{Sceletium}}~: Plusieurs espèces du genre sont utilisées traditionnellement, mais pas de manière interchangeable. Certaines communautés distinguent des « chimiotypes » sur la base d'effets subjectifs différents --- une distinction qui trouve aujourd'hui confirmation dans les analyses chromatographiques révélant des profils d'alcaloïdes distincts entre populations \parencite{krstenansky2016}.
\end{itemize}

\subsubsection{Caractéristique 2~: Maîtrise des Transformations Biochimiques (Fermentation)}

Le processus traditionnel de préparation du Kanna implique une fermentation contrôlée~: les parties aériennes récoltées sont écrasées, humidifiées, et laissées à fermenter dans des sacs en peau ou des récipients hermétiques pendant plusieurs jours à plusieurs semaines. Cette pratique révèle en réalité une \textbf{ingénierie biochimique sophistiquée}.

Les analyses UPLC-MS récentes \parencite{chen2019} démontrent que la fermentation induit des transformations enzymatiques qui~:

\begin{enumerate}
\item Augmentent la concentration en mésembrine (alcaloïde principal) de 300-700\,\% par rapport au matériel frais
\item Réduisent la concentration en mésembrénone (alcaloïde moins stable)
\item Génèrent des métabolites secondaires par dégradation oxydative contrôlée
\end{enumerate}

Ces transformations ne sont pas accidentelles, mais résultent d'un savoir-faire transmis intergénérationnellement, incluant le contrôle de paramètres critiques (température, humidité, durée, aération). La pharmacopée traditionnelle Khoisan avait donc identifié empiriquement des procédés de bioconversion que la biochimie moderne ne commencera à élucider qu'au XXI\textsuperscript{e} siècle.

\subsubsection{Caractéristique 3~: Intégration Holiste des Dimensions Contextuelles}

Les usages traditionnels du Kanna ne se réduisent pas à l'administration d'une substance isolée, mais intègrent des dimensions rituelles, sociales, et symboliques~:

\begin{itemize}
\item \textbf{Contexte rituel}~: Dans certaines communautés, le Kanna est consommé lors de cérémonies collectives structurées, où les effets psychoactifs sont interprétés et intégrés socialement (facilitation de la parole, réduction de l'anxiété sociale, renforcement de la cohésion communautaire).

\item \textbf{Associations synergiques}~: Le Kanna est fréquemment associé à d'autres plantes médicinales (\textit{Leonotis leonurus}, \textit{Hoodia gordonii}) selon des protocoles empiriques dont les bases pharmacologiques n'ont pas été étudiées.

\item \textbf{Adaptabilité individuelle}~: Les guérisseurs traditionnels ajustent les doses et les modes d'administration en fonction des caractéristiques individuelles du patient --- une pratique qui anticipe les principes de la « médecine personnalisée » revendiquée par la pharmacogénomique contemporaine.
\end{itemize}

Ces caractéristiques révèlent que \textbf{le savoir Khoisan sur le Kanna constitue un système épistémologique cohérent et validé}, non un ensemble de « croyances » fortuites.

\section{Épistémicide et Violence Épistémique~: L'Érosion Coloniale des Savoirs Khoisan}

\subsection{Le Concept d'Épistémicide (Boaventura de Sousa Santos)}

Boaventura de Sousa Santos \parencite{santos2014} introduit le concept d'\textbf{épistémicide}~: la destruction systématique, souvent violente, de traditions intellectuelles non-occidentales par leur délégitimation comme « superstitions », « croyances », ou « folklore » --- en opposition aux « connaissances universelles » de la modernité européenne.

L'épistémicide ne désigne pas la perte passive de connaissances, mais un \textbf{processus actif de destruction intellectuelle}, intrinsèquement lié à l'expansion coloniale européenne. Santos identifie trois mécanismes principaux~:

\subsubsection{Mécanisme 1~: La Ligne Abyssale (\textit{Abyssal Thinking})}

Santos postule que la modernité occidentale repose sur une division épistémologique radicale --- une « ligne abyssale » --- séparant deux domaines~:

\begin{itemize}
\item \textbf{Côté métropolitain}~: Le domaine du « vrai » et du « faux », où les débats scientifiques sont reconnus comme légitimes.

\item \textbf{Côté colonial}~: Le domaine de l'« incompréhensible », du « primitif » --- un espace où les traditions intellectuelles autochtones sont disqualifiées comme non-pertinentes pour la production de connaissance légitime.
\end{itemize}

Dans le contexte du Kanna, cette ligne abyssale opère de manière exemplaire~: les débats scientifiques sur les mécanismes d'action de la mésembrine sont reconnus comme des controverses légitimes, tandis que les savoirs Khoisan sur les « chimiotypes » ou les effets de la fermentation ne sont pas traités comme des hypothèses à tester, mais comme des « anecdotes ethnographiques ».

\subsubsection{Mécanisme 2~: L'Extractivisme Épistémique}

Santos analyse l'épistémicide comme forme d'\textbf{extractivisme intellectuel}, analogue structurellement à l'extractivisme économique colonial. Ce processus opère en trois étapes~:

\begin{enumerate}
\item \textbf{Extraction}~: Les chercheurs occidentaux collectent des savoirs ethnobotaniques (souvent sans consentement explicite).

\item \textbf{Traduction/réduction}~: Ces savoirs sont « traduits » dans le langage de la science moderne (identification taxonomique, isolement de molécules actives).

\item \textbf{Appropriation}~: Les résultats sont publiés dans des revues académiques occidentales et brevetés par des entreprises pharmaceutiques, sans reconnaissance des contributeurs intellectuels autochtones.
\end{enumerate}

Le cas Phytopharm/Pfizer (1996-2003) illustre cet extractivisme épistémique~: l'entreprise britannique Phytopharm a développé un extrait standardisé de \kanna{} (Zembrin®) et déposé des brevets sur les alcaloïdes de mésembrine, \textbf{sans consultation ni compensation des communautés Khoisan}. Ce n'est qu'après mobilisation des conseils San et Khoi que des accords de partage des avantages seront négociés rétroactivement \parencite{wynberg2009}.

\subsubsection{Mécanisme 3~: La « Monoculture du Savoir »}

Santos identifie un troisième mécanisme~: l'imposition de la rationalité scientifique moderne comme seule forme légitime de connaissance, entraînant l'invisibilisation de toute épistémologie alternative. Cette monoculture opère par des mécanismes institutionnels subtils~:

\begin{itemize}
\item \textbf{Exclusion des curricula académiques}~: Les savoirs autochtones ne sont pas enseignés dans les universités (sauf dans des départements marginaux d'anthropologie).

\item \textbf{Critères de validation scientifique inaccessibles}~: Les méthodologies de validation biomédicale (essais cliniques randomisés) requièrent des ressources hors de portée des communautés autochtones, créant une asymétrie structurelle.

\item \textbf{Dépendance épistémique}~: Les communautés autochtones finissent par intérioriser l'idée que leurs propres savoirs sont « inférieurs », et se tournent vers les institutions biomédicales pour valider leurs connaissances --- une forme de \textbf{colonialité épistémique} \parencite{quijano2000}.
\end{itemize}

\subsection{L'Érosion Historique des Savoirs Khoisan sur le Kanna}

L'épistémicide désigne des processus historiques concrets. Dans le cas des savoirs Khoisan sur le Kanna, plusieurs vecteurs d'érosion peuvent être identifiés~:

\subsubsection{Vecteur 1~: Génocide Démographique et Dépossession Territoriale}

La colonisation européenne de l'Afrique australe (initiée en 1652) a entraîné une chute démographique catastrophique des populations Khoisan~:

\begin{itemize}
\item \textbf{Guerres coloniales}~: Les conflits armés entre colons et communautés Khoisan ont décimé des populations entières.

\item \textbf{Maladies infectieuses importées}~: La variole (épidémies de 1713, 1755, 1767) a tué jusqu'à 90\,\% de certaines populations Khoekhoe côtières.

\item \textbf{Dépossession territoriale}~: L'expansion des colonies agricoles européennes a repoussé les communautés Khoisan vers des terres marginales, coupant l'accès aux zones de récolte traditionnelles de \kanna{}.
\end{itemize}

\subsubsection{Vecteur 2~: Christianisation et Stigmatisation}

Les missions chrétiennes ont joué un rôle central dans la délégitimation des pratiques médicales Khoisan~:

\begin{itemize}
\item \textbf{Diabolisation des guérisseurs traditionnels}~: Les guérisseurs ont été qualifiés de « sorciers » et leurs pratiques interdites.

\item \textbf{Imposition du modèle médical occidental}~: Les missions établissaient des dispensaires où seule la médecine occidentale était pratiquée, créant une dépendance institutionnelle.

\item \textbf{Destruction des rituels traditionnels}~: Les cérémonies collectives impliquant l'usage de plantes psychoactives ont été interdites comme « païennes ».
\end{itemize}

\subsubsection{Vecteur 3~: Fragmentation Sociale et Prolétarisation}

L'économie coloniale (mines de diamants et d'or) a déstructuré les modes de vie traditionnels Khoisan~:

\begin{itemize}
\item \textbf{Prolétarisation forcée}~: Les communautés ont été contraintes de fournir une main-d'œuvre salariée, abandonnant les pratiques de subsistance qui structuraient la transmission des savoirs botaniques.

\item \textbf{Urbanisation}~: Les migrations vers les townships urbains ont coupé les nouvelles générations de l'accès aux environnements écologiques où les savoirs étaient transmis.

\item \textbf{Scolarisation occidentale}~: Les systèmes scolaires coloniaux ont marginalisé la transmission intergénérationnelle des savoirs autochtones.
\end{itemize}

\subsubsection{Conséquences Contemporaines}

Aujourd'hui, les savoirs Khoisan sur le Kanna sont fragmentés et menacés~:

\begin{itemize}
\item \textbf{Perte de continuité}~: De nombreuses communautés ne pratiquent plus l'usage traditionnel du Kanna.

\item \textbf{Réappropriation commerciale}~: Le Kanna est aujourd'hui principalement commercialisé sous forme de compléments alimentaires standardisés, sans participation significative des communautés Khoisan.

\item \textbf{Folklorisation}~: Les savoirs Khoisan sont parfois mobilisés comme « patrimoine culturel », mais rarement reconnus comme systèmes de connaissance actuels.
\end{itemize}

\section{Vers une Décolonisation Épistémologique~: Méthodologies Participatives}

\subsection{Le Paradigme de la Décolonisation Méthodologique}

Linda Tuhiwai Smith \parencite{smith1999} formule une critique radicale de la recherche académique menée sur les peuples autochtones. Son argument central~: \textbf{La recherche scientifique a historiquement fonctionné comme un instrument d'oppression coloniale}, et toute recherche contemporaine impliquant des communautés autochtones doit être radicalement réformée.

Smith documente comment la recherche académique (anthropologie, linguistique, médecine, botanique) a participé au projet colonial en produisant des savoirs au service de l'administration coloniale, en objectifiant les peuples autochtones, et en extrayant des connaissances sans réciprocité.

Face à cette histoire, Smith formule des principes méthodologiques pour une \textbf{recherche décoloniale}~:

\subsubsection{Principe 1~: Consentement Libre, Préalable et Éclairé (\fpic{})}

Le \textit{Free, Prior and Informed Consent} (\fpic{}), codifié dans la Déclaration des Nations Unies sur les droits des peuples autochtones (2007), impose que toute recherche impliquant des communautés autochtones soit précédée d'un consentement explicite, obtenu sans coercition, avant le début de la recherche, et sur la base d'une information complète.

Dans le contexte de l'ethnopharmacognosie du Kanna, le \fpic{} implique~:

\begin{itemize}
\item \textbf{Consultation préalable}~: Avant toute collecte de plantes ou de savoirs, les chercheurs doivent obtenir l'autorisation des autorités communautaires légitimes (conseils traditionnels San et Khoi).

\item \textbf{Information transparente}~: Les communautés doivent être informées non seulement des objectifs scientifiques, mais aussi des implications commerciales potentielles (développement de compléments alimentaires brevetables).

\item \textbf{Droit de refus}~: Les communautés doivent pouvoir refuser la recherche sans subir de pressions.

\item \textbf{Droit de retrait}~: Les communautés doivent pouvoir retirer leur consentement à tout moment, avec obligation pour les chercheurs de cesser les travaux.
\end{itemize}

\subsubsection{Principe 2~: Co-Production des Protocoles de Recherche}

Smith insiste sur le fait que le \fpic{}, bien que nécessaire, est insuffisant~: il réduit les communautés autochtones à un rôle de « validation rétroactive », sans leur donner de pouvoir décisionnel sur les orientations de recherche.

Une décolonisation méthodologique authentique implique que les communautés autochtones soient intégrées comme \textbf{co-chercheuses} dès la conception des projets, avec pouvoir décisionnel sur~:

\begin{itemize}
\item \textbf{Les questions de recherche}~: Quels aspects du Kanna méritent d'être étudiés en priorité~? La réponse peut différer radicalement selon les perspectives.

\item \textbf{Les méthodologies}~: Quels types d'évidence sont considérés comme légitimes~? Une approche décoloniale devrait intégrer des méthodologies qualitatives (récits narratifs, témoignages) aux côtés des méthodologies quantitatives.

\item \textbf{La publication et diffusion des résultats}~: Qui contrôle la publication~? Les communautés doivent avoir un droit de veto sur la publication de connaissances sensibles, et un droit de co-signature.
\end{itemize}

\subsubsection{Principe 3~: Partage Équitable des Avantages (\textsc{abs})}

Le Protocole de Nagoya (2010, entrée en vigueur 2014), instrument juridique international ratifié par 138 États, impose un cadre de partage des avantages issus de l'utilisation des ressources génétiques et des savoirs traditionnels associés. Ce cadre repose sur trois piliers~:

\begin{enumerate}
\item \textbf{Accès réglementé}~: L'accès aux ressources génétiques et aux savoirs traditionnels est soumis à autorisation préalable.

\item \textbf{Conditions mutuellement convenues (\textsc{mat})}~: Les modalités d'accès et de partage des avantages sont négociées entre les parties.

\item \textbf{Avantages monétaires et non-monétaires}~: royalties sur les ventes de produits dérivés, financements de projets de développement communautaire, co-propriété intellectuelle, transferts de technologie, formations, emplois locaux.
\end{enumerate}

Toutefois, l'analyse critique menée par Wynberg \& Chennells \parencite{wynberg2009}, Robinson \parencite{robinson2010}, et Dutfield \parencite{dutfield2009} révèle des \textbf{lacunes structurelles} du Protocole de Nagoya~:

\begin{itemize}
\item \textbf{Définition ambiguë des « savoirs traditionnels »}~: Le texte ne précise pas si les savoirs doivent être « secrets » pour être protégés.

\item \textbf{Absence de mécanismes de recours efficaces}~: Les communautés autochtones victimes de biopiraterie disposent de peu de moyens juridiques pour poursuivre les entreprises violant le Protocole.

\item \textbf{Faible capacité d'application}~: Beaucoup d'États africains manquent de cadres législatifs nationaux pour opérationnaliser le Protocole.
\end{itemize}

\subsection{Co-Production des Savoirs et Recherche-Action Participative (\textsc{par}/\textsc{cbpr})}

Au-delà des cadres juridiques, des modèles de recherche collaborative ont été développés pour opérationnaliser la décolonisation méthodologique. Le plus influent est la \textbf{Participatory Action Research (\textsc{par})} ou \textbf{Community-Based Participatory Research (\textsc{cbpr})}, issue des travaux de Paulo Freire et adaptée au contexte de la recherche en santé publique.

La \textsc{par}/\textsc{cbpr} repose sur plusieurs principes structurants~:

\begin{description}
\item[Principe 1~: Horizontalité des rapports de pouvoir] \hfill \\
Les communautés possèdent des connaissances expertes sur leurs propres réalités, et doivent être reconnues comme partenaires à part entière.

\item[Principe 2~: Orientation vers l'action] \hfill \\
La \textsc{par} ne vise pas seulement la production de connaissances académiques, mais la transformation des conditions sociales, économiques, ou sanitaires des communautés.

\item[Principe 3~: Itérativité et réflexivité] \hfill \\
La \textsc{par} est structurée en cycles itératifs (planification → action → observation → réflexion → nouvelle planification).
\end{description}

\subsubsection{Application au Kanna~: Un Modèle de Co-Production}

Appliquée à l'ethnopharmacognosie du Kanna, la \textsc{par}/\textsc{cbpr} impliquerait~:

\begin{enumerate}
\item \textbf{Phase 1~: Co-définition des priorités}
   \begin{itemize}
   \item Organisation d'ateliers participatifs avec les conseils San et Khoi pour identifier les questions de recherche prioritaires.
   \end{itemize}

\item \textbf{Phase 2~: Formation de chercheurs communautaires}
   \begin{itemize}
   \item Formations des membres de la communauté aux méthodologies de collecte de données (entretiens semi-structurés, inventaires botaniques, analyses phytochimiques de base).
   \end{itemize}

\item \textbf{Phase 3~: Co-analyse et interprétation}
   \begin{itemize}
   \item Ateliers d'analyse collective des données, où les catégories d'interprétation sont négociées entre perspectives autochtones et biomédicales.
   \end{itemize}

\item \textbf{Phase 4~: Restitution et action}
   \begin{itemize}
   \item Publication conjointe des résultats (co-signature), avec versions accessibles en langues locales. Élaboration de recommandations politiques (réglementation de la récolte commerciale, certification \textsc{abs}).
   \end{itemize}
\end{enumerate}

\subsubsection{Exemple de réussite~: Le Modèle Hoodia}

Un exemple partiel de co-production réussie est l'accord négocié entre le \textsc{csir} (Council for Scientific and Industrial Research, Afrique du Sud) et les communautés San concernant \textit{Hoodia gordonii}, plante coupe-faim dont le principe actif (p57) a été breveté par le \textsc{csir} et licencié à Pfizer.

Après mobilisation des \textsc{ong} et des conseils San dénonçant la biopiraterie, un accord a été signé en 2003 prévoyant~:
\begin{itemize}
\item Versement de 6\,\% des royalties aux communautés San
\item Création d'un trust pour gérer les fonds
\item Représentation San dans les instances de gouvernance
\item Reconnaissance publique de l'origine autochtone des savoirs
\end{itemize}

Toutefois, cet accord a été critiqué pour ses limites~:
\begin{itemize}
\item Négociation \textit{ex post} (après le dépôt de brevets)
\item Absence de co-propriété intellectuelle (brevets restent propriété du \textsc{csir})
\item Royalties versées uniquement si commercialisation réussie (le projet Hoodia a finalement échoué commercialement)
\end{itemize}

Ces limites illustrent que les cadres juridiques actuels restent insuffisants et doivent être complétés par des transformations plus profondes des pratiques de recherche.

\section{Conclusion~: Reconnaissance de la Légitimité Épistémologique}

L'analyse épistémologique déployée dans ce chapitre conduit à une conclusion radicale~: \textbf{Une ethnopharmacologie équitable du Kanna n'est possible qu'à condition de reconnaître les savoirs Khoisan non comme « folklore » ou « pistes de recherche » à exploiter, mais comme systèmes de connaissance à part entière, porteurs de droits moraux et juridiques}.

Cette reconnaissance implique plusieurs ruptures avec les pratiques actuelles de la recherche ethnopharmacologique~:

\begin{enumerate}
\item \textbf{Rupture épistémologique}~: Abandonner le monopole du réductionnisme moléculaire comme seul régime de validation thérapeutique légitime, et intégrer des formes d'évidence qualitatives, narratives, et contextuelles valorisées dans les traditions autochtones.

\item \textbf{Rupture méthodologique}~: Passer d'une logique extractiviste à une logique de co-production (intégration des communautés autochtones comme co-chercheuses dès la conception des projets).

\item \textbf{Rupture institutionnelle}~: Réformer les mécanismes de financement de la recherche, les critères d'évaluation académique (reconnaissance de la co-signature), et les modes de publication (accès ouvert, versions en langues locales).

\item \textbf{Rupture juridique}~: Renforcer les cadres de propriété intellectuelle collective, instaurer des mécanismes de recours efficaces en cas de biopiraterie, et garantir la souveraineté épistémique des communautés autochtones.
\end{enumerate}

Ces ruptures ne visent pas à abolir la recherche biomédicale sur le Kanna, mais à la réorganiser de manière éthiquement soutenable et épistémologiquement rigoureuse. Les chapitres suivants exploreront comment cette reconnaissance épistémologique se décline concrètement dans les domaines de la phytochimie (Chapitre 2), de l'ethnobotanique (Chapitre 3), de la pharmacologie (Chapitre 4), et de la gouvernance (Chapitres 7-9).

\textit{[Transition vers Chapitre 2~: Caractérisation Botanique, Phytochimique et Biogéographique]}

\textit{Après avoir établi la légitimité épistémologique des savoirs Khoisan sur le Kanna, nous pouvons à présent examiner les fondations matérielles de ces savoirs~: la botanique de} \kanna{}\textit{, la composition phytochimique de ses alcaloïdes, et la distribution biogéographique des populations sauvages. Ce chapitre montrera comment les variations écologiques et chimotypiques de la plante valident matériellement les connaissances traditionnelles Khoisan, qui avaient identifié empiriquement ces variations plusieurs siècles avant les analyses chromatographiques modernes.}

% ============================================================================
% BIBLIOGRAPHIE
% ============================================================================

\printbibliography[heading=bibintoc,title={Bibliographie}]

\end{document}
